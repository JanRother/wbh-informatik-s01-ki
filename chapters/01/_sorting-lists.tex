\chapter{Sortieren von Listen}
\label{ch:01_sorting-lists}

\begin{universityTask}
    Ein einfacher Sortieralgorithmus ist wie folgt definiert:  
    Eine unsortierte Liste wird durch die Wahl eines Trennelements in zwei Hälften unterteilt. Alle Elemente, die kleiner als das Trennelement sind, werden in der linken Liste gespeichert; alle, die größer sind, in der rechten. Mit den beiden Teillisten \q{links} und \q{rechts} wird das Verfahren dann erneut durchgeführt. Dies geschieht solange, bis die Liste sortiert ist.

    \begin{enumerate}[label=(\arabic*)]
        \item Wann bricht der Algorithmus ab?
        \item Implementieren Sie den Sortieralgorithmus in PROLOG.
    \end{enumerate}
    
    \universityPoints{20}
\end{universityTask}

Dieser Text enthält die Lösung der obigen Aufgabenstellung.